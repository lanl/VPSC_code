c *** pressure tube texture
c ***  axial=3  hoop=2   radial=1
    JW 899 (H190M)
B   144
    -79.6095      88.4362       0.0000           0.00140532
    -80.2469      85.9186       0.0000           0.00150196
    -81.5297      83.5064       0.0000           0.00163953
    -83.4671      81.6013       0.0000           0.00170688
    -85.8391      80.1860       0.0000           0.00173995
    -88.4169      79.5277       0.0000           0.00176898
    -66.0786      86.4807       0.0000           0.00703716
    -67.7548      81.0361       0.0000           0.00721939
    -71.2319      76.3132       0.0000           0.00571918
    -75.9506      72.7662       0.0000           0.00394477
    -81.1985      70.2214       0.0000           0.00309099
    -86.6467      68.9194       0.0000           0.00267741
    -51.8301      84.8816       0.0000           0.02011804
    -54.5709      77.1326       0.0000           0.01052102
    -59.3850      69.4673       0.0000           0.00299635
    -65.5214      62.2071       0.0000           0.00136964
    -74.0073      57.2735       0.0000           0.00080263
    -83.8371      54.5036       0.0000           0.00061153
    -37.3148      84.0975       0.0000           0.03778594
    -40.0781      74.1125       0.0000           0.00663025
    -44.5562      62.0417       0.0000           0.00130649
    -51.5757      52.2616       0.0000           0.00055862
    -63.5116      43.8546       0.0000           0.00034067
    -79.2730      38.1282       0.0000           0.00026844
    -22.6509      84.1802       0.0000           0.05338791
    -25.0373      71.5079       0.0000           0.00290405
    -28.0016      56.7398       0.0000           0.00074568
    -33.6079      43.2417       0.0000           0.00043675
    -46.5713      31.4583       0.0000           0.00041436
    -69.8563      24.2212       0.0000           0.00033508
     -8.3703      84.5731       0.0000           0.06454319
     -9.2336      69.5567       0.0000           0.00174880
    -10.6538      53.2627       0.0000           0.00065902
    -13.3934      37.8945       0.0000           0.00067159
    -19.3042      25.0281       0.0000           0.00070781
    -44.3096      10.8586       0.0000           0.00068461
    -79.6095      91.5668       0.0000           0.00140532
    -80.2469      94.0845       0.0000           0.00150196
    -81.5297      96.4966       0.0000           0.00163953
    -83.4671      98.4017       0.0000           0.00170688
    -85.8391      99.8170       0.0000           0.00173995
    -88.4169     100.4753       0.0000           0.00176898
    -66.0786      93.5224       0.0000           0.00703716
    -67.7548      98.9669       0.0000           0.00721939
    -71.2319     103.6898       0.0000           0.00571918
    -75.9506     107.2368       0.0000           0.00394477
    -81.1985     109.7816       0.0000           0.00309099
    -86.6467     111.0836       0.0000           0.00267741
    -51.8301      95.1214       0.0000           0.02011804
    -54.5709     102.8704       0.0000           0.01052102
    -59.3850     110.5357       0.0000           0.00299635
    -65.5214     117.7959       0.0000           0.00136964
    -74.0073     122.7295       0.0000           0.00080263
    -83.8371     125.4994       0.0000           0.00061153
    -37.3148      95.9055       0.0000           0.03778594
    -40.0781     105.8905       0.0000           0.00663025
    -44.5562     117.9613       0.0000           0.00130649
    -51.5757     127.7414       0.0000           0.00055862
    -63.5116     136.1484       0.0000           0.00034067
    -79.2730     141.8748       0.0000           0.00026844
    -22.6509      95.8228       0.0000           0.05338791
    -25.0373     108.4951       0.0000           0.00290405
    -28.0016     123.2633       0.0000           0.00074568
    -33.6079     136.7613       0.0000           0.00043675
    -46.5713     148.5447       0.0000           0.00041436
    -69.8563     155.7818       0.0000           0.00033508
     -8.3703      95.4299       0.0000           0.06454319
     -9.2336     110.4463       0.0000           0.00174880
    -10.6538     126.7403       0.0000           0.00065902
    -13.3934     142.1085       0.0000           0.00067159
    -19.3042     154.9750       0.0000           0.00070781
    -44.3096     169.1445       0.0000           0.00068461
     79.6095      88.4362       0.0000           0.00140532
     80.2469      85.9186       0.0000           0.00150196
     81.5297      83.5064       0.0000           0.00163953
     83.4671      81.6013       0.0000           0.00170688
     85.8391      80.1860       0.0000           0.00173995
     88.4169      79.5277       0.0000           0.00176898
     66.0786      86.4807       0.0000           0.00703716
     67.7548      81.0361       0.0000           0.00721939
     71.2319      76.3132       0.0000           0.00571918
     75.9506      72.7662       0.00000           0.00394477
     81.1985      70.2214       0.00000           0.00309099
     86.6467      68.9194       0.00000           0.00267741
     51.8301      84.8816       0.00000           0.02011804
     54.5709      77.1326       0.00000           0.01052102
     59.3850      69.4673       0.00000           0.00299635
     65.5214      62.2071       0.00000           0.00136964
     74.0073      57.2735       0.00000           0.00080263
     83.8371      54.5036       0.00000           0.00061153
     37.3148      84.0975       0.00000           0.03778594
     40.0781      74.1125       0.00000           0.00663025
     44.5562      62.0417       0.00000           0.00130649
     51.5757      52.2616       0.00000           0.00055862
     63.5116      43.8546       0.00000           0.00034067
     79.2730      38.1282       0.00000           0.00026844
     22.6509      84.1802       0.00000           0.05338791
     25.0373      71.5079       0.00000           0.00290405
     28.0016      56.7398       0.00000           0.00074568
     33.6079      43.2417       0.00000           0.00043675
     46.5713      31.4583       0.00000           0.00041436
     69.8563      24.2212       0.00000           0.00033508
      8.3703      84.5731       0.0000           0.06454319
      9.2336      69.5567       0.0000           0.00174880
     10.6538      53.2627       0.0000           0.00065902
     13.3934      37.8945       0.0000           0.00067159
     19.3042      25.0281       0.0000           0.00070781
     44.3096      10.8586       0.0000           0.00068461
     79.6095      91.5668       0.0000           0.00140532
     80.2469      94.0845       0.0000           0.00150196
     81.5297      96.4966       0.0000           0.00163953
     83.4671      98.4017       0.0000           0.00170688
     85.8391      99.8170       0.0000           0.00173995
     88.4169     100.4753       0.0000           0.00176898
     66.0786      93.5224       0.0000           0.00703716
     67.7548      98.9669       0.0000           0.00721939
     71.2319     103.6898       0.0000           0.00571918
     75.9506     107.2368       0.00000           0.00394477
     81.1985     109.7816       0.00000           0.00309099
     86.6467     111.0836       0.00000           0.00267741
     51.8301      95.1214       0.00000           0.02011804
     54.5709     102.8704       0.00000           0.01052102
     59.3850     110.5357       0.00000           0.00299635
     65.5214     117.7959       0.00000           0.00136964
     74.0073     122.7295       0.00000           0.00080263
     83.8371     125.4994       0.00000           0.00061153
     37.3148      95.9055       0.00000           0.03778594
     40.0781     105.8905       0.00000           0.00663025
     44.5562     117.9613       0.00000           0.00130649
     51.5757     127.7414       0.00000           0.00055862
     63.5116     136.1484       0.00000           0.00034067
     79.2730     141.8748       0.00000           0.00026844
     22.6509      95.8228       0.00000           0.05338791
     25.0373     108.4951       0.00000           0.00290405
     28.0016     123.2633       0.00000           0.00074568
     33.6079     136.7613       0.00000           0.00043675
     46.5713     148.5447       0.00000           0.00041436
     69.8563     155.7818       0.00000           0.00033508
      8.3703      95.4299       0.0000           0.06454319
      9.2336     110.4463       0.0000           0.00174880
     10.6538     126.7403       0.0000           0.00065902
     13.3934     142.1085       0.0000           0.00067159
     19.3042     154.9750       0.0000           0.00070781
     44.3096     169.1445       0.0000           0.00068461